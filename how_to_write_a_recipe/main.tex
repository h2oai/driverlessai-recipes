\documentclass[11pt,
			   %10pt, 
               %hyperref={colorlinks},
               aspectratio=169,
               hyperref={colorlinks}
               ]{beamer}
\usetheme{Singapore}
\usecolortheme[snowy, cautious]{owl}

\usepackage[utf8]{inputenc}
\usepackage[T1]{fontenc}
\usepackage[american]{babel}
\usepackage{graphicx}
\usepackage{hyperref}
\hypersetup{
    colorlinks=true,
    urlcolor=[rgb]{1,0,1},
    linkcolor=[rgb]{1,0,1}}

\usepackage[natbib=true,style=authoryear,backend=bibtex,useprefix=true]{biblatex}
\usepackage{listings}
\lstset{numbers=right, 
        numberstyle=\tiny, 
%        breaklines=true,
%        backgroundcolor=\color{light-gray},
        numbersep=5pt,
	xleftmargin=\parindent,
	xrightmargin=.25in} 

%\setbeamercolor*{bibliography entry title}{fg=black}
%\setbeamercolor*{bibliography entry location}{fg=black}
%\setbeamercolor*{bibliography entry note}{fg=black}
\definecolor{OwlGreen}{RGB}{75,0,130} % easier to see
\setbeamertemplate{bibliography item}{}
\setbeamerfont{caption}{size=\footnotesize}
\setbeamertemplate{frametitle continuation}{}
\setcounter{tocdepth}{1}
\renewcommand*{\bibfont}{\scriptsize}
\addbibresource{bibliography.bib}

\renewcommand*{\thefootnote}{\fnsymbol{footnote}}

%\author{\copyright\hspace{1pt}Ashrith Barthur\footnote{\tiny{This material is shared under a \href{https://creativecommons.org/licenses/by/4.0/deed.ast}{CC By 4.0 license} which allows for editing and redistribution, even for commercial purposes. However, any derivative work should attribute the author and H2O.AI.}}}
\author{Ashrith Barthur}
\title{How to Write A Recipe?}
\subtitle{Automating Feature Engineering Using DriverlessAI}
\logo{\includegraphics[height=8pt]{img/h2o_logo.png}}
\institute{\href{https://www.h2o.ai}{H\textsubscript{2}O.ai}}
\date{\today}
\subject{How to Write A Recipe?}

\begin{document}
	
	\maketitle
	
%	\begin{frame}
	
%		\frametitle{Contents}
		
%		\tableofcontents{}
		
%	\end{frame}

%-------------------------------------------------------------------------------
	\subsection{Question}
%-------------------------------------------------------------------------------
	\begin{frame}
		\frametitle{Question}
		\begin{enumerate}
			\item How many of us have built variables, features, transformers, or feature transformers?
			\item What are they?
		\end{enumerate}
	\end{frame}
%-------------------------------------------------------------------------------
	\subsection{Answer}
%-------------------------------------------------------------------------------
	\begin{frame}
		\frametitle{Answer}
		\begin{enumerate}
			\item Variables, features, transformers, feature transformers are refer to the same.
			\item Each column in your data is considered a variable or a feature.
			\item Each \textit{new} column created is also referred to as a variable or a feature.
			\item The process of creating a new variable, or a feature is called a transformation. 
			\item The code processing an \textit{existing} column to a \textit{new} column is called a \textit{transformer}.
		\end{enumerate}
	\end{frame}
%-------------------------------------------------------------------------------
	\subsection{Example Transformation}
%-------------------------------------------------------------------------------
	\begin{frame}[fragile]
		\frametitle{Example Transformation}
		\begin{enumerate}
			\item \textit{height}- Variable
			\item New variable after transformation \textit{log2(height)}
		\end{enumerate}
\end{frame}
%-------------------------------------------------------------------------------
	\subsection{Question}
%-------------------------------------------------------------------------------
	\begin{frame}
		\frametitle{Question}
		\begin{enumerate}
			\item How many of us are familiar with Custom Transformers in Driverless AI?
			\item What are they?
		\end{enumerate}
	\end{frame}
%-------------------------------------------------------------------------------
	\subsection{Answer}
%-------------------------------------------------------------------------------
	\begin{frame}
		\frametitle{Answer}
		\begin{enumerate}
			\item DriverlessAI already has a large, comprehensive set of transformers. 
			\item But there are always domains that require nuanced features. 
			\item And for this, DriverlessAI provides us to create custom transformers. 
			\item This is provided by provisioning an extension class \textit{CustomTransformer}
		\end{enumerate}
	\end{frame}
%-------------------------------------------------------------------------------
	\subsection{Recipe Introduction}
%-------------------------------------------------------------------------------
	\begin{frame}[fragile]
		\frametitle{How Did We Build A Custom Transformer?}
		Driverless AI provides an extension. \\
		This is a class `CustomTransformer`
		\begin{verbatim}
		class ExampleLogTransformer(CustomTransformer):
		\end{verbatim}
\end{frame}
%-------------------------------------------------------------------------------
		\section{Feature Recipe Structure}
%-------------------------------------------------------------------------------
	\begin{frame}[fragile]
		\frametitle{How Did We Build This?}
		The class has:
		\begin{enumerate}
			\item Parameters that need to be provided. 
			\item These parameters are specific to the type of feature recipe that you are building. 
			\item It also has four methods which primarily handle your feature engineering transformation. 
		\end{enumerate}
			
\end{frame}
%-------------------------------------------------------------------------------
		\subsection{Parameters - Basic}
%-------------------------------------------------------------------------------
	\begin{frame}[fragile]
		\frametitle{Parameters - Basic}
		\begin{verbatim}
		class ExampleLogTransformer(CustomTransformer):
			_regression = True
			_binary = True
			_multiclass = True
		\end{verbatim}
			
\end{frame}
%-------------------------------------------------------------------------------
		\subsection{Parameters - Advanced}
%-------------------------------------------------------------------------------
	\begin{frame}[fragile]
		\frametitle{Parameters - Advanced}
		\begin{verbatim}
		class ExampleLogTransformer(CustomTransformer):
			_regression = True
			_binary = True
			_multiclass = True
			_numeric_output = True
			_is_reproducible = True
			_excluded_model_classes = ['tensorflow']
			_modules_needed_by_name = ["custom_package==1.0.0"]
		\end{verbatim}
			
\end{frame}
%-------------------------------------------------------------------------------
%		\subsection{Parameters - Advanced}
%-------------------------------------------------------------------------------
	\begin{frame}[fragile]
		\frametitle{Acceptance Method}
		\begin{verbatim}
		class ExampleLogTransformer(CustomTransformer):
		_regression = True
		_binary = True
		_multiclass = True
		_numeric_output = True
		_is_reproducible = True
		_excluded_model_classes = ['tensorflow']
		_modules_needed_by_name = ["custom_package==1.0.0"]
	
		@staticmethod
		def do_acceptance_test():
		return True
		\end{verbatim}
			
\end{frame}
%-------------------------------------------------------------------------------
%		\subsection{Parameters - Advanced}
%-------------------------------------------------------------------------------
	\begin{frame}[fragile]
		\frametitle{Input Data}
		\begin{verbatim}
		...
		@staticmethod
		def do_acceptance_test():
		return True

		@staticmethod
		def get_default_properties():
		return dict(col_type = "numeric", min_cols = 1, max_cols = 1, 
		relative_importance = 1)
		\end{verbatim}
\end{frame}

%-------------------------------------------------------------------------------
%		\subsection{Parameters - Advanced}
%-------------------------------------------------------------------------------
	\begin{frame}[fragile]
		\frametitle{Input Data Types}
		\begin{verbatim}
		a. "all"         - all column types
		b. "any"         - any column types
		c. "numeric"     - numeric int/float column
		d. "categorical" - string/int/float column considered a categorical for 
		feature engineering
		e. "numcat"      - allow both numeric or categorical
		f. "datetime"    - string or int column with raw datetime such as 
		'%Y/%m/%d %H:%M:%S' or '%Y%m%d%H%M'
		\end{verbatim}
\end{frame}
%-------------------------------------------------------------------------------
%		\subsection{Parameters - Advanced}
%-------------------------------------------------------------------------------
	\begin{frame}[fragile]
		\frametitle{Input Data Types}
		\begin{verbatim}
		g. "date"        - string or int column with raw date such as 
		'%Y/%m/%d' or '%Y%m%d'
		h. "text"        - string column containing text 
		(and hence not treated as categorical)
		i. "time_column" - the time column specified at the start of 
		the experiment (unmodified)
		\end{verbatim}
\end{frame}

%-------------------------------------------------------------------------------
%		\subsection{Parameters - Advanced}
%-------------------------------------------------------------------------------
	\begin{frame}[fragile]
		\frametitle{Fit Function}
		\begin{verbatim}
		@staticmethod
		def get_default_properties():
		return dict(col_type = "numeric", min_cols = 1, max_cols = 1,
		relative_importance = 1)

		def fit_transform(self, X: dt.Frame, y: np.array = None):
			X_pandas = X.to_pandas()
			X_p_log = np.log10(X_pandas)
			return X_p_log
		\end{verbatim}
\end{frame}
%-------------------------------------------------------------------------------
%		\subsection{Parameters - Advanced}
%-------------------------------------------------------------------------------
	\begin{frame}[fragile]
		\frametitle{Transform Function}
		\begin{verbatim}
		def fit_transform(self, X: dt.Frame, y: np.array = None):
			X_pandas = X.to_pandas()
			X_p_log = np.log10(X_pandas)
			return X_p_log

		def transform(self, X: dt.Frame):
			X_pandas = X.to_pandas()
			X_p_log = np.log10(X_pandas)
			return X_p_log
		\end{verbatim}
\end{frame}
%-------------------------------------------------------------------------------
%		\subsection{Parameters - Advanced}
%-------------------------------------------------------------------------------
	\begin{frame}[fragile]
		\frametitle{Library}
		\begin{verbatim}
		from h2oaicore.systemutils import segfault, loggerinfo, main_logger
		from h2oaicore.transformer_utils import CustomTransformer
		import datatable as dt
		import numpy as np
		import pandas as pd
		import logging
		\end{verbatim}
\end{frame}
%-------------------------------------------------------------------------------
%		\subsection{Question}
%-------------------------------------------------------------------------------
	\begin{frame}
		\frametitle{DEMO}
	\end{frame}

%-------------------------------------------------------------------------------
%		\subsection{Question}
%-------------------------------------------------------------------------------
	\begin{frame}
		\frametitle{Advantages}
		\begin{enumerate}
			\item Feature engineering process standardised by:
				\begin{enumerate}
					\item preset parameters
					\item preset methods
				\end{enumerate}
			\item Effort minimisation leads to minimisation in time spent.
			\item Build only once - Feature engineering is carried over from training/testing to production.
			\item DAI automatically, runs multiple models on various sets of features to get the best model. 
			\item All the requirements are handled internally by DAI. 
		\end{enumerate}
\end{frame}
%-------------------------------------------------------------------------------
%	References
%-------------------------------------------------------------------------------

	\begin{frame}[t, allowframebreaks]
	
		\frametitle{References}	
		
			\textbf{How to build a recipe}\\
			\small{\url{https://github.com/ashrith/how_to_write_a_recipe}}
			
		\framebreak		
		
		\printbibliography
		
	\end{frame}
%-------------------------------------------------------------------------------
	\section{Questions}
%------------------------------------------------------------------------------

		\begin{frame}

			\frametitle{Thanks \& Questions}

		\end{frame}

%-------------------------------------------------------------------------------
	%	\section{End}
%------------------------------------------------------------------------------
\end{document}
\end{document}
